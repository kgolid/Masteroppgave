% !TEX root= ../main.tex
\section{Concluding remarks}
\label{sec:Concluding remarks}
In the process of repeatedly generalizing our graph structural definitions, we have reached a situation where our current definition is almost identical to the actual axioms and rule-applications in our proof system; more specifically, the applications with binary NAND-clauses in their conclusion.
This comes as no surprise, but is a bit disappointing.
The original goal was to find some graph structural relation $V$, independent of the definition of Neg, such that for any graph $G$ with vertices $a,b$: $V(a,b) \; \Leftrightarrow \; G\vdash \ol{ab}$ (implication (1) and (2)).
This would as a result give us a graph structure present only when the graph in question was kernel free.
It has however become apparent that no simple graph structural definition suffices in satisfying these implications.
Even $V_3$ falls short in this endeavor.

The big lesson here might be that the existence of small substructures in graphs is not always sufficient in predicting the absence of kernels.
Trying to recognize and isolate inconsistent parts of a graph seems to be the wrong approach in many cases.
We will therefore at this point terminate our search for such a graph structure.

However, in the process of searching for these structures, we developed the counter-example disproving implication (2) of $V_3$ (Figure~\ref{fig:v3_counter_graph}).
This graph will in the next chapter be utilized to disprove the main hypothesis given for this thesis.
