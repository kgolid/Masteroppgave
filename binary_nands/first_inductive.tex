% !TEX root= ../main.tex
\externaldocument{odd_paths}
\section{Generalizing trimming}
\label{sec:Generalizing trimming}
In this section, we will take a closer look at the current concept of trimming, and through some examples introduce a more general definition.
So far, trimming has meant forcing certain vertices to point only at its successor in a path.

In Section~\ref{sec:Odd paths} we motivated the concept of trimmed paths by presenting the Figure~\ref{fig:kernel_untrimmed}.
The figure shows how a path that normally provides a provable NAND-clause may be "ruined" by adding out-edges to certain vertices in the path, making certain vital OR-clauses non-binary and thus making the NAND-clause unprovable.

Notice however that adding a loop on the vertex to which the path branches, leaves us with a completely different situation.\par
\begin{figure}[!h]
  \centering
  \begin{tikzpicture}
    [
    point/.style={circle,draw,inner sep=0pt,minimum size=2mm},
    one/.style={fill=black},
    ]
    \node (1) at (0,1) [point, label=above:$a$] {};
    \node (2) at (1,1) [point, label=above:$x$] {};
    \node (2b) [point, below right=4mm and 6mm of 2, label=below:$z$] {};
    \node (3) at (2,1) [point, label=above:$y$] {};
    \node (4) at (3,1) [point,label=above:$b$] {};
    \draw [-latex] (1) to (2);
    \draw [-latex] (2) to (3);
    \draw [-latex] (2) to (2b);
    \draw [-latex] (3) to (4);
    \draw [-latex, loop right] (2b) to (2b);

    \node (e4) [above right=4mm and 6mm of 4] {};
    \node (e5) [right=8mm of 4] {};
    \node (e6) [below right=4mm and 6mm of 4] {};
    \draw [dashed] (e4) to (4);
    \draw [dashed] (e5) to (4);
    \draw [dashed] (e6) to (4);
  \end{tikzpicture}
  \caption{}
  \label{fig:path_3_with_loop}
\end{figure}
The addition of the loop adds $\ol{z}$ to the axioms.
The NAND-clause $\ol{ab}$ can now easily be proven in the following way:\par
\begin{figure}[!h]
  \centering
  \begin{prooftree*}
    \Hypo{\ol{ax}}
    \Hypo{\ol{yb}}
    \Hypo{\ol{z}}
    \Infer[left label=$xyz$]3{\ol{ab}}
  \end{prooftree*}
  \caption{}
  \label{fig:proof_loop}
\end{figure}

This shows the fact that vertices at odd positions in a path can indeed branch off to other vertices than their path successor and still contribute to a provable NAND-clause, just as long as the vertices they branch off to is provably false in Neg.
If the vertex is provably false, we can use that unary NAND-clause, like we did in the above proof, to handle the non-binary OR-clauses.
This observation lets us generalize our notion of trimmed vertices.

The big problem with this generalization is that it makes our vel-relation no longer a purely graph-structural one.
With such a definition, whenever we want to check whether two vertices have a vel between them, we may have to work out the actual provability of certain unary NAND-clauses.
This is exactly what we try to avoid with our vel-relation.
