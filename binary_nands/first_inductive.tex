% !TEX root= ../main.tex
\externaldocument{odd_paths}
\externaldocument{odd_vels}
\section{Generalizing trimming}
\label{sec:Generalizing trimming}
In this section, we will take a closer look at the current concept of trimming, and through some examples introduce a more general definition.
So far, trimming has meant forcing certain vertices to point only at its successor in a path.

In Section~\ref{sec:Odd paths} we motivated the concept of trimmed paths by presenting the Figure~\ref{fig:kernel_untrimmed}.
The figure shows how a path that normally provides a provable NAND-clause may be "ruined" by adding out-edges to certain vertices in the path, making certain vital OR-clauses non-binary and thus making the NAND-clause unprovable.

Notice however that adding a loop on the vertex to which the path branches, leaves us with a completely different situation.\par
\begin{figure}[!h]
  \centering
  \begin{tikzpicture}
    [
    point/.style={circle,draw,inner sep=0pt,minimum size=2mm},
    one/.style={fill=black},
    ]
    \node (1) at (0,1) [point, label=above:$a$] {};
    \node (2) at (1,1) [point, label=above:$x$] {};
    \node (2b) at (1,0) [point, label=below:$z$] {};
    \node (3) at (2,1) [point, label=above:$y$] {};
    \node (4) at (3,1) [point,label=above:$b$] {};
    \draw [-latex] (1) to (2);
    \draw [-latex] (2) to (3);
    \draw [-latex] (2) to (2b);
    \draw [-latex] (3) to (4);
    \draw [-latex, loop right] (2b) to (2b);

    \node (e4) [above right=4mm and 6mm of 4] {};
    \node (e5) [right=8mm of 4] {};
    \node (e6) [below right=4mm and 6mm of 4] {};
    \draw [dashed] (e4) to (4);
    \draw [dashed] (e5) to (4);
    \draw [dashed] (e6) to (4);
  \end{tikzpicture}
  \caption{}
  \label{fig:path_3_with_loop}
\end{figure}
The addition of the loop adds $\ol{z}$ to the axioms.
The NAND-clause $\ol{ab}$ can now easily be proven in the following way:\par
\begin{figure}[!h]
  \centering
  \begin{prooftree*}
    \Hypo{\ol{ax}}
    \Hypo{\ol{yb}}
    \Hypo{\ol{z}}
    \Infer[left label=$xyz$]3{\ol{ab}}
  \end{prooftree*}
  \caption{}
  \label{fig:proof_loop}
\end{figure}
This shows the fact that vertices at odd positions in a path can indeed branch off to other vertices than their path successor and still contribute to a provable NAND-clause, just as long as the vertices they branch off to are provably false in Neg.
If the vertex is provably false, we can use that unary NAND-clause, like we did in the above proof, to handle the non-binary OR-clauses.
This observation lets us generalize our notion of trimmed vertices.

The big problem with this generalization is that it makes our vel-relation no longer a purely graph-structural one.
With such a definition, whenever we want to check whether two vertices have a vel between them, we may have to work out the actual provability of certain unary NAND-clauses.
This is exactly what we try to avoid with our vel-relation.

To make matters worse, consider the following two graphs:\par
\begin{figure}[h!]
  \centering
  \begin{tikzpicture}
    [
    point/.style={circle,draw,inner sep=0pt,minimum size=2mm},
    one/.style={fill=black},
    ]
    \node (1) at (0,1) [point, label=above:$a$] {};
    \node (2) at (1,1) [point, label=above:$x$] {};
    \node (2b) at (1,0) [point, label=below:$z$] {};
    \node (3) at (2,1) [point, label=above:$y$] {};
    \node (4) at (3,1) [point,label=above:$b$] {};
    \draw [-latex] (1) to (2);
    \draw [-latex] (2) to (3);
    \draw [-latex] (2) to (2b);
    \draw [-latex] (3) to (4);
    \draw [-latex] (2b) to (1);

    \node (e4) [above right=4mm and 6mm of 4] {};
    \node (e5) [right=8mm of 4] {};
    \node (e6) [below right=4mm and 6mm of 4] {};
    \draw [dashed] (e4) to (4);
    \draw [dashed] (e5) to (4);
    \draw [dashed] (e6) to (4);
  \end{tikzpicture}
  \hspace{5mm}
  \begin{tikzpicture}
    [
    point/.style={circle,draw,inner sep=0pt,minimum size=2mm},
    one/.style={fill=black},
    ]
    \node (1) at (0,1) [point, label=above:$a$] {};
    \node (2) at (1,1) [point, label=above:$x$] {};
    \node (2b) at (1,0) [point, label=below:$z$] {};
    \node (3) at (2,1) [point, label=above:$y$] {};
    \node (4) at (3,1) [point,label=above:$b$] {};
    \draw [-latex] (1) to (2);
    \draw [-latex] (2) to (3);
    \draw [-latex] (2) to (2b);
    \draw [-latex] (3) to (4);
    \draw [-latex] (2b) to (4);

    \node (e4) [above right=4mm and 6mm of 4] {};
    \node (e5) [right=8mm of 4] {};
    \node (e6) [below right=4mm and 6mm of 4] {};
    \draw [dashed] (e4) to (4);
    \draw [dashed] (e5) to (4);
    \draw [dashed] (e6) to (4);
  \end{tikzpicture}
  \caption{}
  \label{fig:path_3_with_backedges}
\end{figure}
Like with the graph from Figure~\ref{fig:path_3_with_loop}, the above graphs are both based on Figure~\ref{fig:kernel_untrimmed}, each with only one edge added.
The left variant has $\ol{az}$ in its axioms, while the right variant has $\ol{bz}$, again making their respective proofs of $\ol{ab}$ almost trivial.\par
\begin{figure}[!h]
  \centering
  \begin{prooftree}
    \Hypo{\ol{ax}}
    \Hypo{\ol{yb}}
    \Hypo{\ol{za}}
    \Infer[left label=$xyz$]3{\ol{ab}}
  \end{prooftree}
  \hspace{5mm}
  \begin{prooftree}
    \Hypo{\ol{ax}}
    \Hypo{\ol{yb}}
    \Hypo{\ol{zb}}
    \Infer[left label=$xyz$]3{\ol{ab}}
  \end{prooftree}
  \caption{}
  \label{fig:proof_loop}
\end{figure}
\FloatBarrier
The above proofs do not rely on any provably false vertex, but rather make use of the fact that $z$ is connected to one of the vertices contained in the NAND-clause that we are trying to prove.

In the case of Figure~\ref{fig:path_3_with_backedges}, $\ol{ab}$ will be provable as long as for all the successors $z$ of $x$, either $\ol{z}$, $\ol{az}$ or $\ol{bz}$ are provable in Neg.

This means that our concept of trimming needs further weakening, taking into account the observation made above.
Such a weakening will simply add the case where, given a vel between $a$ and $b$, trimmed vertices may branch off to vertices $c$ as long as either $\ol{ac}$ or $\ol{bc}$ are provable in Neg.

The definition now bases the provability of $\ol{ab}$ on the provability of a set of other binary NAND-clauses, suggesting an inductive approach for our definitions.

\section{Inductive definitions}
\label{sec:Inductive definitions}
All the graph structures presented in this chapter so far can easily be defined inductively.
For instance, given some graph $\mathbf{G}=(G,N)$, let $V_1 \subseteq G \times G$ denote the set of all pairs of vertices related by our original vel-definition (Definition~\ref{def:vel}) where `trimmed' meant strictly non-branching.

We can define $V_1$ inductively in the following way, with $a$ and $b$ being arbitrary vertices from $G$.
\begin{align}
  \textbf{(BC):}&\quad N(a,b) \vee N(b,a) &\Rightarrow  V_1(a,b)\\
  \textbf{(IS):}&\quad \exists c \in N(a): (N(c) =\{d\} \wedge V_1(d,b)) &\Rightarrow V_1(a,b)
\end{align}
The symmetry in the base case is what makes this a definition of a vel and not a path, while the trimmed-ness gets expressed by the restrictions we set on vertex $c$.

Adding the new weaker trimming-restriction is now easy.
Let $V_2 \subseteq G \times G$ be the set of all pairs of vertices related by our new vel-definition.
$V_2$ can now be defined inductively in the following way, again with $a$ and $b$ as arbitrary vertices from $G$.
\begin{align}
  \textbf{(BC):}&\quad N(a,b) \vee N(b,a) &\Rightarrow V_2(a,b)\\
  \textbf{(IS):}&\quad \exists c \in N(a):
  \left ( \begin{tabular}{l}
  $\exists d \in N(c): V_2(d,b) \quad \wedge$\\
  $\forall d \in N(c): V_2(d,b) \vee V_2(d,a) \vee V_2(d,d)$
  \end{tabular} \right )
  &\Rightarrow V_2(a,b)
\end{align}
