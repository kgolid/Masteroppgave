% !TEX root= ../main.tex
\section{Unary and binary NAND-clauses in graphs}
\label{sec:Unary and binary NAND-clauses in graphs}
\subsection{Inconsistency and unary NAND-clauses}
\label{sub:Inconsistency and unary NAND-clauses}
Since our proof system, Neg, only has one rule, the last step of an inconsistency proof will always look the same:
\begin{prooftree*}
  \Hypo{\ol{x_1}}
  \Hypo{\ol{x_2}}
  \Hypo{\ol{x_3}}
  \Hypo{\dots}
  \Infer4[$x_1x_2x_3\dots$]{\varnothing}
\end{prooftree*}
The premise will always consist of a collection of NAND-clauses, each of length 1, together with an OR-clause equal to the union of all the NAND-clauses.
It is easy to see that none of the NAND-clauses can be larger than unary, since that would result in a non-empty NAND-clause in the conclusion.
The OR-clause has to equal the union of the NAND-clauses simply by definition of the (RNeg)-rule.

This fact was also observed and formalized by Walicki in \cite{michal-completeness}:
\begin{align}
  \Gamma \vdash_{Neg} \{\} \Leftrightarrow \exists K \in \text{OR}: (\forall  k \in K: \Gamma \vdash_{Neg} \ol{k})
\end{align}
We know from the definition that any OR-clause used in the proof system corresponds to a single vertex with its successors in the graph.
We do however not know what the NAND-clauses of length 1 might correspond to.
Knowing this would, by soundness and completeness of Neg, give us a complete picture of the graph structure needed for a kernel not to exist.\todo{Currently brushing over the restrictions set to the proofs of completeness and soundness. Will fix this later.}

The only thing we \textit{do} know about unary NAND-clauses is that they correspond to vertices that is assigned 0 in all models of the graph.
We get this from soundness of Neg.
This is however not the graph structural property we are ultimately looking for, but we can at least say that we have reduced the question ``What does an inconsistent graph look like?'' to the question ``What does a provably false vertex look like?''.

So what does a unary NAND-clause proven in Neg correspond to in the graph?
Similarly to a proof of $\varnothing$, there is really just one way to prove a unary NAND-clause:
\begin{prooftree*}
  \Hypo{\ol{xy_1}}
  \Hypo{\ol{xy_2}}
  \Hypo{\dots}
  \Hypo{\ol{y_n}}
  \Hypo{\ol{y_{n+1}}}
  \Hypo{\dots}
  \Infer6[$y_1y_2\dots y_ny_{n+1}\dots$]{\ol{x}}
\end{prooftree*}
Any derivation of a unary NAND-clause $\ol{x}$ must end with a rule application using $K \in \text{OR}$ where the NAND-clauses in the premises are either unary, $\ol{k}$ or binary, $\ol{xk}$, both such that $k \in K$.

We formalize this observation in the following way:
\begin{align}
  \Gamma \vdash_{Neg} \ol{x} \Leftrightarrow \exists K \in \text{OR}: (\forall k \in K: \Gamma \vdash_{Neg} \ol{kx} \vee \Gamma \vdash_{Neg} \ol{k}) \wedge (\exists k \in K: \Gamma \vdash_{Neg} \ol{kx})
\end{align}
Just as we reduced the problem of inconsistency to the problem of unary NAND-clauses, we are able to reduce the problem further to binary NAND-clauses.
We could even continue the reduction further to ternary clauses, quaternary clauses and so on, but without a change of strategy at some point, this seems pointless.

There might actually be a good reason for changing our strategy when reaching the problem of describing \textit{binary} NAND-clauses.
Observe that all Neg-proofs shown in this thesis has used unary and binary NAND-clauses only. 
