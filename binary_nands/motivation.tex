% !TEX root= ../main.tex
\section{Motivation}
\label{sec:Motivation}
Since our proof system, Neg, has only one rule, the last step of any inconsistency proof will always look the same:\par
\begin{figure}[!h]
  \centering
  \begin{prooftree*}
    \Hypo{\ol{x_1}}
    \Hypo{\ol{x_2}}
    \Hypo{\ol{x_3}}
    \Hypo{\dots}
    \Infer4[$x_1x_2x_3\dots$]{\varnothing}
  \end{prooftree*}
  \caption{}
  \label{fig:proof_unary_nand}
\end{figure}
The premise will always consist of a collection of NAND-clauses, each of length 1, together with an OR-clause equal to the union of all the NAND-clauses.
It is easy to see that none of the NAND-clauses can be larger than unary, since that would result in a non-empty NAND-clause in the conclusion.
The OR-clause has to equal the union of the NAND-clauses simply by definition of the (RNeg)-rule.

This fact was also observed and formalized by Walicki\cite{michal-completeness}:
\begin{align}
  \Gamma \vdash \{\} \Leftrightarrow \exists K \in \text{OR}: (\forall  k \in K: \Gamma \vdash \ol{k})
\end{align}
We know from the definition that any OR-clause used in the proof system corresponds to a single vertex with its successors in the graph.
We do however not know what the NAND-clauses of length 1 might correspond to.
Knowing this would, by soundness and completeness of Neg, give us the graph structure needed for a kernel not to exist.

The only thing we \textit{do} know about unary NAND-clauses is that they correspond to vertices that are assigned 0 in all models of the graph.
We get this from soundness of Neg.
This is however not the graph structural property we are ultimately looking for, but we can at least say that we have reduced the question ``What does an inconsistent graph look like?'' to the question ``What does a provably false vertex look like?''.

So what does a unary NAND-clause proven in Neg correspond to in the graph?
Similarly to a proof of $\varnothing$, there is really just one way to prove a unary NAND-clause:\par
\begin{figure}[!h]
  \centering
  \begin{prooftree*}
    \Hypo{\ol{xy_1}}
    \Hypo{\ol{xy_2}}
    \Hypo{\dots}
    \Hypo{\ol{y_n}}
    \Hypo{\ol{y_{n+1}}}
    \Hypo{\dots}
    \Infer6[$y_1y_2\dots y_ny_{n+1}\dots$]{\ol{x}}
  \end{prooftree*}
  \caption{}
  \label{fig:proof_unary_nand}
\end{figure}
Any derivation of a unary NAND-clause $\ol{x}$ must end with a rule application using $K \in \text{OR}$ where for each $k \in K$, there is a NAND-clause in the premise that is either unary, $\ol{k}$ or binary, $\ol{xk}$.
In addition, we require the premise to contain at least one binary NAND-clause, in order to actually be able to conclude with $\ol x$ and not $\varnothing$.

We formalize this observation in the following way:
\begin{align}
  \Gamma \vdash \ol{x} \Leftrightarrow \exists K \in \text{OR}:
  \left ( \begin{tabular}{l}
  $\exists k \in K: G \vdash \ol{kx} \; \wedge$\\
  $\forall k \in K: G \vdash \ol{kx} \vee G \vdash \ol{k}$
  \end{tabular} \right )
\end{align}
Just as we reduced the problem of inconsistency to the problem of unary NAND-clauses, we are able to reduce the problem further to binary NAND-clauses.
We could even continue the reduction further to ternary clauses, quaternary clauses and so on, but without a change of strategy at some point, this seems pointless.

Our current reduction lets us ask how two vertices $x,y$ are \textit{connected} in the graph when their binary NAND $\ol{xy}$ is proven in Neg.
Observe that we have parts of this correspondence down already, with our axioms being all binary.
Vertices directly connected by an edge must therefore be a part of the corresponding graph structure.
%Finding some graph structural relation that corresponds to binary NAND-clauses might also give us some knowledge about a potential standard form of binary-NAND-proofs.\todo{does this make sense?}

Let $R(a,b)$ denote the existence of some graph structure $R$ between the vertices $a$ and $b$ in some graph $G$; our correspondence can now be formally defined as follows:
\begin{align}
  R(a,b) \quad \Rightarrow \quad G \vdash \ol{ab}\\
  R(a,b) \quad \Leftarrow \quad G \vdash \ol{ab}
\end{align}
The two implications will be referred to as \textit{implication (1)} and \textit{implication (2)}, respectively.

Suppose we manage to find a structure $R$ that satisfied both above implications.
The following equation tells us how that would directly give us a predicate deciding whether or not the graph has a kernel.
\begin{align}
  &Sol(G) = \varnothing\\
  \Leftrightarrow\; &G \vdash \varnothing\\
  \overset{2.1}{\Leftrightarrow}\;&\exists K \in \text{OR}: (\forall  k \in K: G \vdash \ol{k})\\
  \overset{2.2}{\Leftrightarrow}\;&\exists K \in \text{OR}: (\forall  k \in K: ( \exists L \in \text{OR}:
  \left ( \begin{tabular}{l}
  $\exists l \in L: G \vdash \ol{lk} \; \wedge$\\
  $\forall l \in L: G \vdash \ol{lk} \vee G \vdash \ol{l}$
  \end{tabular} \right ) ) )\\
  \overset{2.3}{\underset{2.4}{\Leftrightarrow}}\;&\exists K \in \text{OR}: (\forall  k \in K: ( \exists L \in \text{OR}:
  \left ( \begin{tabular}{l}
  $\exists l \in L: G \vdash R(l,k) \; \wedge$\\
  $\forall l \in L: G \vdash R(l,k) \vee G \vdash R(l,l)$
  \end{tabular} \right ) ) )
\end{align}
The following sections will present various graph structures, each satisfying implication (1).
Each presented structure will be a generalization of the previous one, ultimately aiming to find a structure general enough to satisfy both implication (1) and implication (2).
Such a structure would be the true graph structural equivalent of a binary NAND provable in Neg.
We did however not manage to find such a graph structure in this thesis.

For each structure presented, implication (1) will be shown, followed by an example disproving implication (2).
The counterexamples will be graphs with the presented structure absent, but where the corresponding NAND-clause is still provable.

Because of soundness of Neg, if two vertices correspond to a provable binary NAND in Neg, for any kernel $K$ in that graph, at least one of the two vertices is outside $K$.
The contrapositive of this observation being that for a graph $G$, if there exists a kernel $K \subseteq G$ such that two vertices, $x_1$ and $x_2$ are both in $K$, then $\ol{x_1x_2}$ is not provable from $G$.
This fact will be used when arguing why some graph structures are \textit{not} satisfying implication (1).
