% !TEX root= ../main.tex
\section{Odd paths}
\label{sec:Odd paths}
Given two vertices $x_1$ and $x_2$ in a graph $G$, if we in Neg are able to prove $\ol{x_1x_2}$ from the axioms we get from $G$, we say that $x_1$ and $x_2$ are \textit{vel-connected}.
This section will try to answer the question ``what kind of graphs structures imply that certain vertices are vel-connected?''.

We already know that whenever two vertices are connected by an edge, like the two vertices in the graph below \footnote{Dashed lines will represent possible edges to irrelevant parts of the graph.
If a vertex has no dashed edges, it means that we disallow any additonal edges going out from that vertex.}, their binary NAND is trivially provable in Neg, since it will be a part of the axioms.

\[
  \begin{tikzpicture}
    [
    point/.style={circle,draw,inner sep=0pt,minimum size=2mm},
    ]
    \node (1) at (1,1) [point] {};
    \node (2) at (2,1) [point] {};
    \draw [-latex] (1) to (2);

    \node (e1) [above left=4mm and 6mm of 1]  {};
    \node (e2) [left=8mm of 1] {};
    \node (e3) [below left=4mm and 6mm of 1] {};
    \node (e4) [above right=4mm and 6mm of 2] {};
    \node (e5) [right=8mm of 2] {};
    \node (e6) [below right=4mm and 6mm of 2] {};
    \draw [dashed] (e1) to (1);
    \draw [dashed] (e2) to (1);
    \draw [dashed] (e3) to (1);
    \draw [dashed] (e4) to (2);
    \draw [dashed] (e5) to (2);
    \draw [dashed] (e6) to (2);
  \end{tikzpicture}
\]
Two vertices are thus vel-connected when one is the successor of the other.
It is however easy to find a case examplifying how the implication does not hold in the opposite direction.
\[
  \begin{tikzpicture}
    [
    point/.style={circle,draw,inner sep=0pt,minimum size=2mm},
    ]
    \node (1) at (0,1) [point,label=above:$x_1$] {};
    \node (2) at (1,1) [point,label=above:$x_2$] {};
    \node (3) at (2,1) [point,label=above:$x_3$] {};
    \node (4) at (3,1) [point,label=above:$x_4$] {};
    \draw [-latex] (1) to (2);
    \draw [-latex] (2) to (3);
    \draw [-latex] (3) to (4);

    \node (e1) [above left=4mm and 6mm of 1]  {};
    \node (e2) [left=8mm of 1] {};
    \node (e3) [below left=4mm and 6mm of 1] {};
    \node (e4) [above right=4mm and 6mm of 4] {};
    \node (e5) [right=8mm of 4] {};
    \node (e6) [below right=4mm and 6mm of 4] {};
    \draw [dashed] (e1) to (1);
    \draw [dashed] (e2) to (1);
    \draw [dashed] (e3) to (1);
    \draw [dashed] (e4) to (4);
    \draw [dashed] (e5) to (4);
    \draw [dashed] (e6) to (4);
  \end{tikzpicture}
\]
The above graph provides us the following axioms:
$NAND = \{\ol{x_1x_2},\ol{x_2x_3},\ol{x_3x_4}\}$, $OR = \{x_1x_2,x_2x_3,x_3x_4\}$.
From these axioms, we can now, despite the fact that the vertices $x_1$ and $x_4$ are not connected by an edge, easily prove the NAND-clause $\ol{x_1x_4}$:.

\begin{prooftree*}
  \Hypo{\ol{x_1x_2}}
  \Hypo{\ol{x_3x_4}}
  \Infer[left label=$x_2x_3$]2{\ol{x_1x_4}}
\end{prooftree*}

Intuitively, one can imagine that the proof above is connecting two NAND-clauses using an OR-clause, resulting in a new binary NAND-clause containing vertices that are weaklier connected than the ones we started with.
This can be done repeatedly, resulting in the ability to prove binary NAND-clauses from vertices that are connected by arbitrarily long paths of the kind above.

Consider for example the following graph:
\[
  \begin{tikzpicture}
    [
    point/.style={circle,draw,inner sep=0pt,minimum size=2mm},
    ]
    \node (1) at (0,1) [point,label=above:$x_1$] {};
    \node (2) at (1,1) [point,label=above:$x_2$] {};
    \node (3) at (2,1) [point,label=above:$x_3$] {};
    \node (4) at (3,1) [point,label=above:$x_4$] {};
    \node (5) at (4,1) [point,label=above:$x_5$] {};
    \node (6) at (5,1) [point,label=above:$x_6$] {};
    \node (7) at (6,1) [point,label=above:$x_7$] {};
    \node (8) at (7,1) [point,label=above:$x_8$] {};
    \node (9) at (8,1) [point,label=above:$x_9$] {};
    \node (10) at (9,1) [point,label=above:$x_{10}$] {};
    \draw [-latex] (1) to (2);
    \draw [-latex] (2) to (3);
    \draw [-latex] (3) to (4);
    \draw [-latex] (4) to (5);
    \draw [-latex] (5) to (6);
    \draw [-latex] (6) to (7);
    \draw [-latex] (7) to (8);
    \draw [-latex] (8) to (9);
    \draw [-latex] (9) to (10);

    \node (e1) [above left=4mm and 6mm of 1]  {};
    \node (e2) [left=8mm of 1] {};
    \node (e3) [below left=4mm and 6mm of 1] {};
    \node (e4) [above right=4mm and 6mm of 10] {};
    \node (e5) [right=8mm of 10] {};
    \node (e6) [below right=4mm and 6mm of 10] {};
    \draw [dashed] (e1) to (1);
    \draw [dashed] (e2) to (1);
    \draw [dashed] (e3) to (1);
    \draw [dashed] (e4) to (10);
    \draw [dashed] (e5) to (10);
    \draw [dashed] (e6) to (10);
  \end{tikzpicture}
\]

$\ol{x_1x_{10}}$ can now be proven in Neg in the following way:
\begin{prooftree*}
  \Hypo{\ol{x_1x_2}}
  \Hypo{\ol{x_3x_4}}
  \Infer[left label=$x_2x_3$]2{\ol{x_1x_4}}
  \Hypo{\ol{x_5x_6}}
  \Infer[left label=$x_4x_5$]2{\ol{x_1x_6}}
  \Hypo{\ol{x_7x_8}}
  \Infer[left label=$x_6x_7$]2{\ol{x_1x_8}}
  \Hypo{\ol{x_9x_{10}}}
  \Infer[left label=$x_8x_9$]2{\ol{x_1x_{10}}}
\end{prooftree*}

Observe that the above proof is also proving $\ol{x_1x_4}$, $\ol{x_1x_6}$ and $\ol{x_1x_8}$ along the way, all in which contain vertices connected by paths of \textit{odd} length.
This is an important point.
NAND-clauses containing vertices connected only by paths of \textit{even} length cannot be proven in the same manner as above.
Usually, such NAND-clauses cannot be proven at all.

Generalizing this, we get that whenever two vertices, $x_1$ and $x_k$, are connected by a fully trimmed path of odd length, they are vel-connected.
Their corresponding NAND-clause $\ol{x_1x_k}$ can be proven in Neg in the following way:
\begin{prooftree*}
  \Hypo{\ol{x_1x_2}}
  \Hypo{\ol{x_3x_4}}
  \Infer[left label=$x_2x_3$]2{\ol{x_1x_4}}
  \Hypo{\ol{x_5x_6}}
  \Infer[left label=$x_4x_5$]2{\ol{x_1x_6}}
  \Ellipsis{}{\ol{x_1x_{k-2}}}
  \Hypo{\ol{x_{k-1}x_k}}
  \Infer[left label=$x_{k-2}x_{k-1}$]2{\ol{x_1x_k}}
\end{prooftree*}
