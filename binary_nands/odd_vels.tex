% !TEX root= ../main.tex
\section{Odd vels}
\label{sec:Odd vels}
We observed that the addition of successors to some vertices in a fully trimmed path made no difference to the proof of the corresponding NAND-clause.
From that observation, we were able to generalize the concept of fully trimmed paths to the concept of oddly trimmed paths, while still having our original implication hold.
In a similar manner, the following observation will further generalize our structure.

The direction of edges in a graph can often be changed individually while still keeping many axiomatic clauses unchanged.
Actually, all axiomatic NAND-clauses stay unchanged under such an operation.

Consider the following graph:

\begin{figure}[!h]
  \centering
  \begin{tikzpicture}
    [
    point/.style={circle,draw,inner sep=0pt,minimum size=2mm}
    ]
    \node (x0) at (0,3) [point, label=above:$x_0$] {};
    \node (x1) at (1,2) [point, label=above:$x_1$] {};
    \node (x2) at (2,1) [point, label=above:$x_2$] {};
    \node (x3) at (3,0) [point, label=above:$x_3$] {};
    \node (x4) at (4,1) [point, label=above:$x_4$] {};
    \node (x5) at (5,2) [point, label=above:$x_5$] {};
    \draw [-latex] (x0) to (x1);
    \draw [-latex] (x1) to (x2);
    \draw [-latex] (x2) to (x3);
    \draw [-latex] (x4) to (x3);
    \draw [-latex] (x5) to (x4);

    \node (e1) [below left=6mm and 4mm of x3]  {};
    \node (e2) [below=8mm of x3] {};
    \node (e3) [below right=6mm and 4mm of x3] {};
    \draw [dashed] (e1) to (x3);
    \draw [dashed] (e2) to (x3);
    \draw [dashed] (e3) to (x3);

    \node (b0) [below left=6mm and 4mm of x0] {};
    \node (b2) [below left=6mm and 4mm of x2] {};
    \node (b5) [below right=6mm and 4mm of x5] {};
    \draw [dashed] (x0) to (b0);
    \draw [dashed] (x2) to (b2);
    \draw [dashed] (x5) to (b5);
  \end{tikzpicture}
  \caption{}
  \label{fig:vel-example}
\end{figure}
From the above graph, $\ol{x_0x_5}$ can be proven in the same way as in the proof of $\ol{x_0x_9}$ from earlier.
Again, the only thing we are changing in terms of the axioms are the OR-clauses that are not used in the proof.

We will call this kind of graph structure an \textit{odd vel} and define it formally in the following way:
Two vertices $a$ and $b$ have an odd vel between them if there exists a vertex $c$ such that there are oddly trimmed paths from $a$ to $c$ and from $b$ to $c$, one of even length (possibly 0) and one of odd length.

Notice that an oddly trimmed path of odd length is just an instance of an odd vel where the even path is of length 0.
