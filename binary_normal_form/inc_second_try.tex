\begin{lemma}
  Given the graph from Figure ... let the vertices $u$ and $v$ be from different components.
  If $\ol{uv}$ is binary-derivable, then its proof contains either $\ol{a^N}$, $\ol{a^W}$ or $\ol{a^E}$.
  \label{thm:uv_proof_contains_a}
\end{lemma}

\begin{proof}
  Base Case:
  No axiom $\ol{uv}$ exists such that $u$ and $v$ are vertices from different components, so our claim vacuously holds.

  Induction Step:
  Suppose we have a binary proof of $\ol{uv}$ where $u$ and $v$ are from different components.
  The immediate premise must contain at least one binary NAND-clause containing $u$ and one containing $v$.
  \begin{figure}[!h]
    \centering
    \begin{prooftree*}
      \Hypo{\ol{u\dots}}
      \Hypo{\ol{v\dots}}
      \Hypo{\dots}
      \Infer[]3{\ol{uv}}
    \end{prooftree*}
    \caption{}
    \label{fig:proof_scheme_uv}
  \end{figure}
  If either $u$ or $v$ are in a clause together with a vertex from a component different from their own, we get from the induction hypothesis that their proof must contain either $a^N$, $a^W$ or $a^E$, so we are done.

  Otherwise, $u$ and $v$ are each either in a clause together with the vertex $s$ or a vertex from their own component.
  Both of these cases results in using the OR-clause $sa^Na^Wa^E$ in the last proof step, since it is the only OR-clause containing $s$ and the only OR-clause containing vertices from different components.
  The premise therefore contains two additional NAND-clauses; one of them containing the $a$-vertex of the component not containing $u$ or $v$ - lets call it $a^w$.

  If the clause containing $a^w$ is unary, we are done.
  If the clause containing $a^w$ is binary, it must either contain $u$ or $v$ in order to give $\ol{uv}$ in the conclusion.
  Since $a^w$ is in a component different from both $u$ and $v$, the induction hypothesis gives us the claim.

  The proof must therefore contain either $a^N$, $a^W$ or $a^E$.
\end{proof}

\begin{lemma}
  $\ol{a^N}$, $\ol{a^W}$ and $\ol{a^E}$ are not binary-derivable.
  \label{thm:non_binary_derivable_a}
\end{lemma}

\begin{proof}
  We prove it by structural induction over the complexity of the proof tree.
  Base Case:
  Neither $\ol{a^N}$, $\ol{a^W}$ nor $\ol{a^E}$ are axioms, so the claim vacuously holds.

  Induction step:
  Suppose we have a binary proof of $\ol{a}$, where $a$ is either $a^N$, $a^W$ or $a^E$.
  Since the proof is binary, we get from our induction hypothesis that neither $\ol{a^N}$, $\ol{a^W}$ nor $\ol{a^E}$ appears in the proof.

  We get from Lemma~\ref{} that $\ol{a}$ is not binary-derivable using clauses from within its own component only, so the binary proof must use vertices outside the component containing $a$; either the $s$-vertex or vertices from other components.

  If it uses $s$, consider the last proof step with $s$ in the premise.
  The OR-clause used in this proof step must be $sa^Na^Wa^E$, being the only OR-clause containing $s$ and thus the only clause able to remove it.
  This OR-clause contains 4 vertices, so the premise must contain 3 additional NAND-clauses;
  one containing $a^N$, one containing $a^W$ and one containing $a^E$.
  We get the following restrictions on these 3 NAND-clauses.
  \begin{itemize}
    \item None of them can be unary, from the induction hypothesis.
    \item None of them can be binary and contain $s$, since that would give an $s$ in the conclusion, contradicting our assumption of this being the last proof step with $s$ in the premise.
    \item None of them can be binary and contain vertices from two different components, from the induction hypothesis and Lemma~\ref{thm:uv_proof_contains_a}.
  \end{itemize}
  The 3 additional NAND-clauses must therefore all contain a second vertex from their own component.
  Since the three component are disjoint, these three vertices are different, making the conclusion of the proof step non-binary, contradicting our original assumption of the proof being binary.

  If the proof does not contain $s$, then it uses some vertex $p$ from another component.
  Consider the last proof step with an ``external'' vertex in the premise.
  The proof step must remove $p$, and conclude with some clause containing vertices only from within the component containing $a$.
  The premise must obviously contain a NAND-clause containing $p$, but it must also contain at least one binary NAND-clause with a vertex from within N, in order to make the conclusion non-empty.
  That clause cannot contain any external vertices, from Lemma 1 and IH, so it must a clause with two N-vertices.
  The OR-clause used must therefore contain the external vertex $p$ and a vertex from N.
  $sa^Na^Wa^E$ is the only one with this property, but we have assumed that $s$ is not in this proof.

  $\ol{a}$ is thus not binary-derivable.

\end{proof}
