% !TEX root= ../main.tex
\externaldocument{unrestricted}
\externaldocument{graph_restricted}
\begin{definition}
  A clause is \textit{binary-derivable} if it is provable in Neg using only binary NAND-clauses.
\end{definition}
This chapter will investigate the following conjecture:
\begin{conjecture}
  Whenever a graph theory is inconsistent, the inconsistency is binary-derivable.
\end{conjecture}
This conjecture was given by supervisor as a main hypothesis for this thesis.

Such a result would be significant both because it would be a strong property for a proof system in general, but also because it could potentially help us to further characterize a kernel-free graph.

Having that any inconsistency can be proven in Neg using unary and binary NAND-clauses only, might imply a similar property in kernel-free graphs, namely that inconsistencies can be described as collections of pairwise structural relations between vertices.

Section~\ref{sec:Unrestricted theories} will disprove a variant of the conjecture, looking at general theories, not necessarily from graphs.
Section~\ref{sec:Graph theories} will prove that some provable binary NAND-clauses are not binary-derivable.
Based on this, some arguments will be made for the case that the actual conjecture also fails.
