% !TEX root= ../main.tex
\subsection{Discourse Theories and Digraphs}
\label{sub:Discourse Theories and Digraphs}
As mentioned earlier, there is a close connection between (1) models of a discourse theory, (2) kernels of a graph and (3) solutions of a graph.
While Roy Cook has given us the equality of (2) and (3), we will now look at two functions connecting (1) and (2).  We get the following definitions from \cite{}

$\mathcal{T}:$ translating a digraph \textbf{G} into a corresponding theory $\mathcal{T}(\mathbf{G})$ such that $sol(\mathbf{G}) = mod(\mathcal{T}(G))$.

$\mathcal{G}:$ translating a theory $T$ into a corresponding digraph $\mathcal{G}(T)$ such that $mod(T) = sol(\mathcal{G}(T))$.
