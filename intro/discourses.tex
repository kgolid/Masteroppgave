% !TEX root= ../main.tex
\externaldocument{../proofs/cnf_to_gnf}
\section{Graph Normal Form}
\label{sec:Graph Normal Form}
A propositional theory over a set of variables $\Sigma$ is in \textit{graph normal form (GNF)}\cite{apal-digraph} if all its formulae have the following form:
\begin{align}
  x \leftrightarrow \bigwedge_{y \in I_x} \neg y
\end{align}
where $I_x \subseteq \Sigma$ and such that every variable occurs exactly once on the left of $\leftrightarrow$ across all the formulae in the theory.

There is a simple translation from a theory in conjunctive normal form to an equisatisfiable theory in graph normal form (shown in Chapter~\ref{sec:Translating CNF to GNF}).
Since conjunctive normal form is expressively complete, we get that any propositional theory, including our labelled statements, has an equisatisfiable GNF theory.

This means that any discourse can be represented with a GNF theory such that the discourse is paradoxical if and only if the GNF theory is inconsistent.
This GNF representation of discourses is interesting to us because GNF theories has a tight correspondence to graphs.
This correspondence lets us not only decide the satisfiability of a discourse theory by looking at certain features in the corresponding graph, but the graph also provide us with the actual models of the discourse, if they exist.

In order to express this logic/graph correspondence, we first need to establish some graph terminology.
