% !TEX root= ../main.tex
\section{Discourses}
\label{sec:Discourses}
A propositional theory is in \textbf{graph normal form (GNF)} if all its formulae have the following form:
\begin{align}
  x \leftrightarrow \bigwedge_{i \in I_x} \neg y_i
\end{align}
such that every variable occurs exactly once on the left of $\leftrightarrow$ across all the formulae in the theory.

There is a simple translation from conjunctive normal from to graph normal form (shown in the appendix), showing that any propositional theory has an equisatisfiable GNF theory.
By interpreting the variable on the left as the name of the statement on the right, like shown earlier, one can start using GNF to model meta-statements.

We now formally define a \textbf{discourse} to be a theory in GNF, and a \textbf{paradox} to simply be an inconsistent discourse.

We will later show a very handy correspondence between these discourse theories and certain graphs.
The correspondence lets us not only decide the satisfiability of a discourse theory (i.e. whether or not it is paradoxical) by looking at certain properties of the corresponding graph.
The properties in the graph also provide us with the satisfying models, if they exist.
In order to express this logic/graph correspondence, we first need to establish some graph terminology.
