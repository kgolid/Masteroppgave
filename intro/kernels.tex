% !TEX root= ../main.tex
\subsection{Graphs, Kernels and Solutions}
\label{sub:Graphs, Kernels and Solution}
A directed graph is a pair \textbf{G} = $\langle G,N \rangle$ where $G$ is a set of vertices while $N \subseteq G \times G$ is a binary relation representing the edges in \textbf{G}.
We use the notation $N(x)$ to denote the set of all vertices that are targeted by edges originating in $x$ (successors of $x$).
Similarly, $N^-(x)$ denotes the set of all vertices with edges targeting $x$ (predecessors of $x$.
We define these two predicates formally as follows:
\begin{align}
  N(x) := \{y \;|\; (x,y) \in N\}\\
  N^-(x) := \{ y \;|\; (y,x) \in N \}
\end{align}
We extend these relations to sets of vertices in the following way:
\begin{align}
  N(X) = \bigcup_{x \in X} N(x)\\
  N^-(X) = \bigcup_{x \in X} N(x)
\end{align}
A kernel is a set of vertices $K \subseteq G$ such that:
\begin{align}
  G \setminus K = N^-(K)
\end{align}
The above equivalence can ble split up into two inclusions to be more easily understood:

$G \setminus K \subseteq N^-(K)$, saying that each vertex outside the kernel, has to have an edge into the kernel (K is dominating/absorbing).

$N^-(K) \subseteq G \setminus K$, saying that each edge targeting a vertex within the kernel has to come from outside, thus no two vertices in the kernel are connected by an edge (K is independent).

We will get the correspondence between satisfying models of a discourse theory and kernels in a graph through an alternative, equivalent kernel definition called a \textit{solution}.
The equivalence of kernels and solutions was shown by Roy Cook in \cite{cook}.

Given a directed graph \textbf{G} = $\langle G,N \rangle$, an assignment $\alpha \in 2^G$ is a function mapping every vertex in the graph to either 0 or 1.
A solution is an assignment $\alpha$ such that for all $x \in G:$
\begin{align}
  \alpha(x) = 1 \iff \alpha(N(x)) = \{ 0 \}
\end{align}
In simple words, this means that for any node $x$, if $x$ is assigned to 1, then all its successors has to be assigned to 0, and if $x$ is assigned to 0, then there has to exist a node assigned to 1 among its successors.
A consequence of this definition is that all sink nodes (nodes with no outgoing edges) in the graph have to be assigned to 1, since it vacuously does not point to any node assigned to 1.
We use the notation $sol(\mathbf{G})$ to denote the set of all solutions of the graph \textbf{G}.
