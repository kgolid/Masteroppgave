% !TEX root= ../main.tex
\section{Graphs, Kernels and Solutions}
\label{sec:Graphs, Kernels and Solution}
A directed graph (digraph) is a pair \textbf{G} = $\langle G,N \rangle$ where $G$ is a set of vertices while $N \subseteq G \times G$ is a binary relation representing the edges in \textbf{G}.
We use the notation $N(x)$ to denote the set of all vertices that are targeted by edges originating in $x$ (successors of $x$).
Similarly, $N^-(x)$ denotes the set of all vertices with edges targeting $x$ (predecessors of $x$).
We define these two functions formally as follows:
\begin{align}
  N(x) := \{y \;|\; (x,y) \in N\}\\
  N^-(x) := \{ y \;|\; (y,x) \in N \}
\end{align}

A simple \textit{path} is a sequence of distinct vertices $x_1,x_2,\dots,x_n$ such that for each consecutive pair $x_i,x_{i+1}$ from the sequence, we have $(x_i, x_{i+1}) \in N$.
We say that two paths are \textit{disjoint} if they do not share any vertices (possibly with the exception of their initial nodes).

The functions $N$ and $N^-$ can be extended pointwise to sets in the following way:
\begin{align}
  N(X) = \bigcup_{x \in X} N(x)\\
  N^-(X) = \bigcup_{x \in X} N(x)
\end{align}

A kernel is a set of vertices $K \subseteq G$ such that:
\begin{align}
  G \setminus K = N^-(K)
\end{align}
The above equivalence can be split into two inclusions to be more easily understood:

$G \setminus K \subseteq N^-(K)$, saying that each vertex outside the kernel has an edge into the kernel (K is \textit{absorbing}).
A consequence of this is that a kernel has to be non-empty, unless the graph is empty.

$N^-(K) \subseteq G \setminus K$, saying that each edge targeting a vertex within the kernel has to come from outside, thus no two vertices in the kernel are connected by an edge (K is \textit{independent}).

Kernels heve been studied over several decades, not only in graph theory, but also within the fields of game theory and economics.
The concept was first defined and used by von Neumann and Morgenstern in \cite{neumann}.
In a graph representing some sort of a turn-based game, where vertices are states and edges are transitions between states, one can often work out winning strategies whenever one finds a kernel in the graph.
Whenever one is outside of the kernel, one always has the possibility of moving inside the kernel (since the kernel is absorbing), while inside the kernel one \textit{has} to move out of it (since the kernel is independent).
If you are the player with the choice outside the kernel, you can control the game and choose to stabilize it by always moving into the kernel, forcing the opponent to move out again on the next turn.

Deciding the existence of kernels in finite graphs has been shown to be an NP-complete problem\cite{chvatal}.
This should not be surprising, since we are in the middle of showing the equivalence between this problem and the problem of finding satisying models of PL theories (SAT), which we know is NP-complete \footnote{We are concerned with SAT over infinitary formulae in this paper, not the finite version from computer science.}.

We will get the correspondence between models of a discourse theory and kernels in a graph through an alternative, equivalent kernel definition called a \textit{solution}.

Given a directed graph \textbf{G} = $\langle G,N \rangle$, an assignment $\alpha \in 2^G$ is a function mapping every vertex in the graph to either 0 or 1.
A \textbf{solution} is an assignment $\alpha$ such that for all $x \in G:$
\begin{align}
  \alpha(x) = 1 \iff \alpha(N(x)) = \{ 0 \}
\end{align}
This means that for any node $x$, if $x$ is assigned 1, then all its successors have to be assigned  0, and if $x$ is assigned 0, then there has to exist a node assigned 1 among its successors.
A consequence of this definition is that all sink nodes (nodes with no outgoing edges) in the graph have to be assigned 1, since it vacuously does not point to any node assigned 1.
We use the notation $sol(\mathbf{G})$ to denote the set of all solutions of the graph \textbf{G}.
