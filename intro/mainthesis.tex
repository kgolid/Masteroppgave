% !TEX root= ../main.tex
\externaldocument{kernelthoery}
\subsection{Dags without kernels}
\label{sub:Dags without kernels}
Knowing that any graph can translate to an equisatisfiable dag, the challenge is now to find sufficient conditions for dags to have kernels, even weaker than the one proved by Richardson (the fact that any finitary dag has a kernel is a direct consequence of Richardson's Theorem).

Michał Walicki has proposed the following thesis:
\begin{align}
  \textit{If a dag has no kernel then it has a ray with infinitely many vertices dominating it.}
\end{align}
Some terminology:
A ray is an infinite sequence of unique vertices $x_0, x_1, x_2, \dots$ such that any two consecutive vertices $x_i,x_i+1$ are connected by the edge$\langle x_i,x_{i+1} \rangle$.
A vertex is dominating a ray if it has infinitely many disjoint paths to it.

The contrapositive of Walicki's thesis suggests a weaker condition for a kernel, since a dag having a ray with infinitely many vertices dominating it implies that the dag is infinitary.
