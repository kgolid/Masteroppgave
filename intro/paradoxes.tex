% !TEX root= ../main.tex
\section{Paradoxes}
\label{sec:Paradoxes}
A theory in propositional logic is semantically \textit{inconsistent} if it has no model, i.e., there exists no variable assignment making all formulae in the theory true.
Consider the following example:
\begin{align}
  x \wedge \neg x
\end{align}
While a sentence like $\neg a \rightarrow b$ can be satisfied by, for instance, letting both $a$ and $b$ be true, no such assignment can be made for the statement above.
The statement is therefore inconsistent.

A paradox is usually informally defined as something along the lines of \textit{``a statement that can be neither true nor false''}.
We can immediately note one thing from this intuitive definition:
Since no paradoxes can be true, all paradoxes are, by definition, inconsistent.
It is however not the case that all inconsistent theories are paradoxes.
Just consider the inconsistent statement, $x \wedge \neg x$ again: this statement simply seems false, and not paradoxical.

A different view is that a paradox is a \textit{dialetheia}, a sentence that is \textit{both} true and false\cite{sep-dialetheism}. We will however not spend much time exploring these philosophical differences, as this is not a philosophical paper and it won't change much for our definitions.

The liar sentence is probably the most famous example of a paradox:
\begin{quote}
  ``This sentence is false.''
\end{quote}
If the statement is true, then the statement is false, but if the statement is false, then the statement is true.
In order to study these kinds of meta-statements, we need a way to reference other statements within a statement.
In propositional logic, we can do this by giving statements "names" in the form of adding fresh variables with equivalences to their correponding statements\todo{bad wording?}.
Consider the left statements below, together with their corresponding named statements on the right.
It can thus neither be true nor false, since both lead to a contradiction.

Notice how the liar sentence is a statement about other statements (in this case itself).
In general, a collection of statements where some of them may refer to themselves or other statements, is often called a \textit{discourse}\cite{synthese-pdl}.
\begin{align}
  a               && x_1 \leftrightarrow a\\
  a \wedge \neg a && x_2 \leftrightarrow a \wedge \neg a\\
  a \vee \neg a   && x_3 \leftrightarrow a \vee \neg a
\end{align}
By performing this naming-operation on these statements, one is obviously changing their truth value.
Even though we have one consistent, one inconsistent and one tautological statement on the left, all the statements become consistent after they have been named.
This is because we in all the cases above can find a truth value for $x_i$ that matches the one of the corresponding statement.
This will not be the case for paradoxes, so our new named statements will be consistent if and only if they are not paradoxical.

Consider the liar sentence.  It can be written as a named statement in the following way:
\begin{align}
  x \leftrightarrow \neg x
\end{align}
This statement is obviously inconsistent, making it a paradox by our newly acquired definition.  The study of \textit{discourses} takes this formalization a step further.
