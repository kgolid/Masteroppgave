% !TEX root= ../main.tex
\section{Paradoxes}
\label{sec:Paradoxes}
A theory in propositional logic is semantically \textit{inconsistent} if it has no model, i.e., there exists no variable assignment making all formulae in the theory true.
Consider the following example:
\begin{align}
  x \wedge \neg x
\end{align}
While a sentence like $\neg a \rightarrow b$ can be satisfied by, for instance, letting both $a$ and $b$ be true, no such assignment can be made for the statement above.
The statement is therefore inconsistent.

A paradox is usually informally defined as something along the lines of \textit{``a statement that can be neither true nor false''}.
We can immediately note one thing from this intuitive definition:
Since no paradoxes can be true, all paradoxes are, by definition, inconsistent.
It is however not the case that all inconsistent theories are paradoxes.
Just consider the inconsistent statement, $x \wedge \neg x$ again: this statement simply seems false, and not paradoxical.

A different view is that a paradox is a \textit{dialetheia}, a sentence that is \textit{both} true and false\cite{sep-dialetheism}. We will however not spend much time exploring these philosophical differences, as this is not a philosophical paper and it won't change much for our definitions.

The liar sentence is probably the most famous example of a paradox:
\begin{quote}
  ``This sentence is false.''
\end{quote}
If the statement is true, then the statement is false, but if the statement is false, then the statement is true.
It can thus neither be true nor false, since both lead to a contradiction.

Notice how the liar sentence is a statement about other statements (in this case itself).
A collection of statements where some of them may refer to themselves or other statements, is called a \textit{discourse} in \cite{synthese-pdl}, which we will follow.
In order to represent such discourses, we need a formal way of referencing other statements within statements.
In propositional logic, this can be done simply by giving statements ``names'' in the form of adding fresh variables with equivalences to their corresponding statements\todo{bad wording?}.

Consider the examples below; with normal propositional statements to the left, together with the corresponding \textit{named} statements to the right.
\begin{align}
  a               && x_1 \leftrightarrow a\\
  a \wedge \neg a && x_2 \leftrightarrow a \wedge \neg a\\
  a \vee \neg a   && x_3 \leftrightarrow a \vee \neg a
\end{align}
Labelling statements in this way obviously changes their truth value.
Even though we have one consistent, one inconsistent and one tautological statement on the left, all the statements become consistent after they have been named.
This is because we can find truth values for both $x_1$, $x_2$ and $x_3$ that match their corresponding statements, making each equivalence true.
In other words, the truth value of a labelled statement does not refer to whether the unnamed statement is consistent, but whether or not a truth value can be found at all.
Since we have defined a paradox to be a statement that is neither true nor false, we get that a statement is paradoxical if and only if labelling it makes it inconsistent.

Consider the liar sentence.
We are able to label it and then use its label in order to reference itself.
Doing this gives us the following result:
\begin{align}
  x \leftrightarrow \neg x
\end{align}
This statement is obviously inconsistent, making it a paradox by our definition.

The correspondence between paradoxical discourses and inconsistent sets of labelled statements motivates us to study labelled statements closer, trying to uncover patterns that prevent inconsistencies and thus paradoxes in the represented discourse.
