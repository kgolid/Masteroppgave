% !TEX root= ../main.tex
\section{Paradoxes}
\label{sec:Paradoxes}
A formula in propositional logic is semantically \textit{inconsistent} if it has no model, i.e. there exists no variable assignment making the formula true.
Likewise, a collection of formulae, called a theory, is inconsistent if there exists no variable assignment making all the formulae in the theory true.
Consider the following formula:
\begin{align}
  x \wedge \neg x
\end{align}
While a sentence like $\neg a \rightarrow b$ can be satisfied by, for instance, letting both $a$ and $b$ be true, no such assignment can be made for the formula above.
The formula is therefore inconsistent.

A paradox can be informally defined as \textit{``a statement that can be neither true nor false''}.
In this case, a statement can be both a formula or a theory.
We can immediately note that since no paradoxes can be true, all paradoxes are, by definition, inconsistent.
It is however not the case that all inconsistencies are paradoxes;
just consider $x \wedge \neg x$ again: this formula simply seems false, and not paradoxical.

A different view is that paradoxes are \textit{dialetheia} -- statements that are \textit{both} true and false\cite{sep-dialetheism}.
We will however not spend much time exploring these philosophical differences, as this is not a philosophical paper and it will not change much for our definitions.

The liar sentence is probably the most famous example of a paradox:
\begin{quote}
  ``This sentence is false.''
\end{quote}
If the sentence is true, then the sentence is false, but if the sentence is false, then the sentence is true.
It can thus neither be true nor false, since both lead to a contradiction.

Notice how the liar sentence is a statement about other statements (in this case itself).
A collection of formulae where some of them may refer to themselves or other formulae is called a \textit{discourse / discourse theory} in \cite{synthese-pdl}, which we will follow.
In order to represent such discourses, we need a formal way of referring to other statements within a statement.
In propositional logic, this can be done by giving statements names.
We name a statement by introducing a bi-implication between it and a fresh variable.

Consider the examples below.
On the left side are normal propositional statements;
on the right side are their corresponding \textit{named} statements, the fresh variables being their names.
\begin{align}
  a               && x_1 \leftrightarrow a\\
  a \wedge \neg a && x_2 \leftrightarrow a \wedge \neg a\\
  a \vee \neg a   && x_3 \leftrightarrow a \vee \neg a
\end{align}
Labelling statements in this way obviously changes their truth value.
Even though there is one consistent, one inconsistent and one tautological statement on the left, all the statements on the right are consistent.
This is because we can find truth values for both $x_1$, $x_2$ and $x_3$ that match the truth value of their corresponding statements, making each equivalence true.
In other words, the truth value of a labelled statement does not refer to whether the unnamed statement is consistent, but whether or not a truth value can be found for it at all.
Since we have defined a paradox to be a statement that is neither true nor false, we get that a statement is paradoxical if and only if labelling it makes it inconsistent.

Consider the liar sentence again.
Labelling it and then using its label in order to make it reference itself gives us the following statement:
\begin{align}
  x \leftrightarrow \neg x
\end{align}
This labelled statement is obviously inconsistent, making it a paradox by our definition.

We continue to look at labelled formulae and ways of determining whether or not they are consistent.
