% !TEX root= ../main.tex
\section{This thesis}
\label{sec:This thesis}
\subsection{Motivation}
\label{sub:Motivation}
Having that (Neg) is both sound and refutationally complete, results linking inconcistency proofs in (Neg) to structures in graphs is of great interest to us, since it can potentially further weaken the current conditions we have for kernels in graphs.
This thesis will thus be concerned mainly with two kinds of questions;
\begin{enumerate}
  \item Are there any characterizing qualities of the proof system (Neg), potentially under certain restrictions?
  \item If the axioms come from a graph, are there any provable results in (Neg) that correspond to certain structures in the graph?
\end{enumerate}
The first question covers qualities like proof length (complexity), size and number of clauses used throughout a proof, the existence of certain clauses in certain proofs, or maybe even the existence of an entire proof normal form.
For instance, an answer to the following question could be of great interest to us: ``Are there certain kinds of clauses that needs to be present in a proof in order to prove $\emptyset$?''.

The second question covers questions like ``What does it mean for the graph that its axioms can prove $\emptyset$?''.
Of course, we know the answer to this question, it means that the graph has no kernel (by soundness of (Neg)).
But what about the provability of $\ol{x}$ for a node $x$ in the graph?
What does that mean for the graph and the node $x$?

Finding a type of clause neccesarily present in an inconsistency proof, together with results on the graph-structural counterpart of that type of clause, gives us in combination a graph-structure neccesarily present in a kernel-free graph (by soundness of (Neg)).
\subsection{Thesis Overview}
\label{sub:Thesis Overview}
THIS SECTION IS NOT UP TO DATE WITH THE FOLLOWING CHAPTERS
Section 2 will take a look at the potential to control the size of NAND-clauses in an inconsistency proof without any restrictions on the axioms.
Section 3 will study the same potential as above, but with graph-based axioms.
We are in both cases above particularly concerned with whether we are able to restrict our proofs to only use binary NAND-clauses.
Section 4 will explore the graph structure that corresponds to unary and binary NAND-clauses, and explain their potential importance.
