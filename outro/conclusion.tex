% !TEX root= ../main.tex
This thesis has found various graph structures implying provability of certain NAND-clauses in Neg.
The process of continuously generalizing these structures has given several examples of unconventional graphs still providing certain provable clauses.
One of these exemplified that some provable binary NAND-clauses are not binary-derivable.
This example was further extended to show that Neg is \textit{not} refutationally complete when restricted to using binary NAND-clauses only.

In our context, this is primarily a negative result.
If BNeg \textit{was} refutationally complete, our search for a graph-structural equivalent of provable clauses would be easier, allowing us to assume that any clause is provable using binary NAND-clauses only.

Knowing this, one can move forward by trying to find other characterizing features of Neg.
One example is its non-explosiveness, mentioned in \cite{michal-completeness}
While more traditional proof systems often are explosive and thus able to prove anything from inconsistent theories, Neg can only prove certain clauses.
It would be interesting to take a closer look at these clauses that are provable by the virtue of the graph being inconsistent, and potentially develop some definition of the ``explosive range'' of a graph.
The concept of \textit{local kernels}, as defined and studied in \cite{synthese-pdl}, might contribute to this definition.

The non-explosiveness of Neg comes from the more general fact that Neg does not have \textit{weakening}, i.e. while rules like $\vdash x \Rightarrow \vdash x \vee y$ are admissible in most proof systems (necessarily so for completeness), this is not the case for Neg.

Since Neg is not complete, it does not hold in general that $G \vDash \ol{A} \Rightarrow G \vdash \ol{A}$.
An interesting question might therefore be ``\textit{when} does it hold''?
Corollary~5.1 from \cite{michal-completeness} tells us that $\Gamma \vDash \ol{A} \Leftrightarrow \exists B \subseteq A: \Gamma \vdash \ol{B}$.
A direct consequence of this is that $\Gamma \vDash \ol{A} \Rightarrow \Gamma \vdash \ol{A}$ holds when $\forall B \subset A: \Gamma \not\vdash B$ and by soundness also when $\forall B \subset A: \Gamma \not\vDash B$
We can formalize this into a corollary on its own:
\begin{corollary}
  Given a graph $\mathbf{G} = \langle G,N \rangle$ and a set of vertices $A \subseteq G$, let $\Gamma = \mathcal{T}(\mathbf{G})$:
  \begin{align}
    \Gamma \vDash \ol{A} \wedge \forall B \subset A: \Gamma \not\vDash \ol{ B } \;\Rightarrow \; \Gamma \vdash \ol{A}
  \end{align}
\end{corollary}
In words, if no kernel in $G$ contains $A$, but for each subset $B \subset A$, there is a kernel that contains $B$, then $\ol{A}$ is provable in Neg.

Also, more work could be put into exploring graph structural relations $V$ such that $G \vdash \ol{ab} \Rightarrow V(a,b)$.
In hindsight, such relations could potentially contribute to Walicki's conjecture in a more direct way than what relations satisfying the inverse implication would.
