% !TEX root= ../main.tex
This thesis has found various graph structures implying provability of certain NAND-clauses in Neg.
The process of continuously generalizing these structures has given several examples of unconventional graphs still providing certain provable clauses.
One of these exemplified that some provable binary NAND-clauses are not binary-derivable.
This example was further extended to show that Neg is \textit{not} refutationally complete when restricted to using binary NAND-clauses only.

In our context, this is primarily a negative result.
If BNeg \textit{was} refutationally complete, our search for a graph-structural equivalent of provable clauses would be easier, allowing us to assume that any clause is provable using binary NAND-clauses only.

Knowing this, one can move forward by trying to find other characterizing features of Neg.
One example is its non-explosiveness, mentioned in \cite{michal-completeness}
While more traditional proof systems often are explosive and thus able to prove anything from inconsistent theories, Neg can only prove certain clauses.
It would be interesting to take a closer look at these clauses that are provable by the virtue of the graph being inconsistent, and potentially develop some definition of the ``explosive range'' of a graph.
Neg is not complete, so it does not hold in general that $G \vDash \ol{A} \Rightarrow G \vdash \ol{A}$, but it might seem like Neg can prove a semantically true NAND-clause $\ol{A}$ when none of its subsets $\ol{B} \subset \ol{A}$ are provable.
This can be formalized as the following conjecture:
\begin{conjecture}
  Given a graph $G$ and a set of vertices $A \subseteq G$:
  \begin{align}
    G \vDash \ol{A} \wedge \forall B \subset A: G \not\vDash \ol{ B } \;\Rightarrow \; \vdash A
  \end{align}
\end{conjecture}

Also, more work could be put into exploring graph structural relations $V$ such that $G \vdash \ol{ab} \Rightarrow V(a,b)$.
In hindsight, such relations could potentially contribute to Walicki's conjecture in a more direct way than what relations satisfying the inverse implication would.
