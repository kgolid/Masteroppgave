% !TEX root= ../main.tex
This thesis has found various graph structures implying provability of certain NAND-clauses in Neg.
The process of continuously generalizing these structures has given several examples of unconventional graphs still providing certain provable clauses.
One of these exemplified that some provable binary NAND-clauses are not binary-derivable.
This example was further extended to show that Neg is \textit{not} refutationally complete when restricted to using binary NAND-clauses only.

In our context, this is primarily a negative result.
If BNeg \textit{was} refutationally complete, our search for a graph-structural equivalent of provable clauses would be easier, allowing us to assume that any clause is provable using binary NAND-clauses only.

Knowing this, one can move forward by trying to find other features of Neg.
One example is its non-explosiveness, mentioned in \cite{michal-completeness}.
While classical proof systems are explosive and thus able to prove anything from inconsistent theories, Neg can only prove certain clauses.
It would be interesting to take a closer look at these clauses that are provable by the virtue of the graph being inconsistent, and potentially develop some definition of the ``explosive range'' of a graph.
Coming back to our original interest in paradoxes, this property becomes useful in attempts to identify and isolate the part of a theory that makes it paradoxical.
The concept of \textit{local kernels}, as defined and studied in \cite{synthese-pdl}, might contribute to this definition of an explosive range.

The non-explosiveness of Neg comes from the more general fact that Neg does not have \textit{weakening}, i.e. while rules like $\Gamma \vdash x \; \Rightarrow \; \Gamma \vdash x \vee y$ are admissible in classical proof systems, this is not the case for Neg.

Since Neg is not complete, it does not hold in general that $\Gamma \vDash \ol{A} \Rightarrow \Gamma \vdash \ol{A}$.
An interesting question might therefore be ``\textit{when} does it hold?''.
Corollary~5.1 from \cite{michal-completeness} tells us that $\Gamma \vDash \ol{A} \Leftrightarrow \exists B \subseteq A: \Gamma \vdash \ol{B}$.
A direct consequence of this is that $\Gamma \vDash \ol{A} \Rightarrow \Gamma \vdash \ol{A}$ holds when $\forall B \subset A: \Gamma \not\vdash B$ and by soundness also when $\forall B \subset A: \Gamma \not\vDash B$.
We can formalize this into a corollary on its own:
\begin{corollary}
  Given a graph $\mathbf{G} = \langle G,N \rangle$ and a set of vertices $A \subseteq G$, let $\Gamma = \mathcal{T}(\mathbf{G})$:
  \begin{align}
    \Gamma \vDash \ol{A} \wedge \forall B \subset A: \Gamma \not\vDash \ol{ B } \;\Rightarrow \; \Gamma \vdash \ol{A}
  \end{align}
\end{corollary}
In words, if no kernel in $G$ contains $A$, but for each subset $B \subset A$, there is a kernel that contains $B$, then $\ol{A}$ is provable in Neg.

Based on this observation, one might be able to find some interesting relations between the kernels in the graph and the set $A$.

Also, more work could be put into exploring graph structural relations $V$ such that for a graph and a subset $A$ of its vertices, we have $\Gamma \vdash \ol{A} \Rightarrow V(A)$.
The relation ``being connected in the underlying undirected graph'' is certainly such a relation, as shown in Section~\ref{sub:Isolated components}.
If one is able to strengthen that relation, it will probably make a bigger contribution to the proof of Walicki's conjecture than relations satisfying the inverse implication.
