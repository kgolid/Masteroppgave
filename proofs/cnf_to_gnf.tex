% !TEX root= ../main.tex
\subsection{Translating CNF to GNF}
\label{sub:Translating CNF to GNF}
Since any PL theory can be expressed in CNF, showing that any theory $P$ in CNF can be translated to a theory $R$ in GNF such that $P$ and $R$ are equisatisfiable gives us that any PL theory has an equisatisfiable GNF.

Given any CNF theory $P$, for each formula in it, follow the steps below to acquire its corresponding GNF theory.

\textbf{Step 1:}
For each literal $\neg x_i$ in the formula, introduce a fresh variable $x'_i$ and add the following two GNF formulae to $P$:
$x'_i \lar \neg x_i, x_i \lar \neg x'_i,$ (unless this has already been done while translating an earlier formula in the theory).

\textbf{Step 2:}
In each clause of the formula, replace every negative literal $\neg x_i$ with its corresponding $x'_i$ fom step 1.
Every clause in the formula does now contain positive literals only.

\textbf{Step 3:}
For every clause $(x_1 \vee x_2 \vee \dots \vee x_n)$, replace it with the following formula, where $y$ is fresh:
\[y \lar (\neg x_1 \wedge \neg x_2 \wedge \dots \wedge \neg x_n \wedge \neg y)\]

This formula is equisatisfiable with our original clause.

The combined set of all these acquired formulae will make up our GNF-theory.
We have only added proper GNF formulae and all variables appear to the left in exactly one clause, so $R$ will indeed be a GNF theory.

Step 1 is adding a bi-implication formula that is always satisfiable, since one of two variables in it is fresh.
One can think of it as a labelling of an already existing variable.
Thus, adding these formulae to a theory does not change its satisfiability/non-satisfiability.

Step 2 is replacing each negative literal with its label, also not changing the satisfiability.

Step 3 is replacing one formula with an equisatisfiable formula, thus not changing the overall satisfiability.

Since none of the steps above change the satisfiability of the theory, we can conclude that our acquired theory is equisatisfiable with to original CNF theory.

Below are some examples of CNF-theories with their corresponding GNF-theories.\\

\begin{example}
  \begin{align}
    \text{CNF:\quad}& ( a \vee b )\\
    \text{GNF:\quad}& ( a \lar \neg a'), (a' \lar \neg a), (b \lar \neg b'), (b' \lar \neg b), (y_1 \lar (\neg a \wedge \neg b \wedge \neg y_1))\\
    \\
    \text{CNF:\quad}& (\neg a)\\
    \text{GNF:\quad}& ( a \lar \neg a'), (a' \lar \neg a), (y_1 \lar (\neg a' \wedge \neg y_1))\\
  \end{align}
\end{example}
