% !TEX root= ../main.tex
\subsubsection{Translating CNF to GNF}
\label{subs:Translating CNF to GNF}
Since any PL theory can be expressed in CNF, showing that any theory $P$ in CNF can be translated to a theory $R$ in GNF such that $P \lar R$ gives us that any PL theory can be expressed in GNF.

Given any CNF theory $P$, start with an empty theory $R$ and for each formula in $P$, follow the steps below to acquire its corresponding GNF formulae.

\textbf{Step 1:}
For each literal $\neg x_i$ in the formula, introduce a fresh variable $x'_i$ and add the following two GNF formulae to $R$: $x'_i \lar \neg x_i, x_i \lar \neg x'_i,$ (unless this has already been done while translating an earlier formula in the theory).

\textbf{Step 2:}
In each clause, replace every negative literal $\neg x_i$ with its corresponding $x'_i$ fom step 1. Every clause does now contain all positive literals.
For every clause $(x_1 \vee x_2 \vee \dots \vee x_n)$, create a fresh variable $y$ and add the following GNF formula to $R$:
\[y \lar (\neg x_1 \wedge \neg x_2 \wedge \dots \wedge \neg x_n \wedge \neg y)\]
The combined set of all these acquired formulae will make up the the corresponding theory $R$.
We have only added proper GNF formulae and all variables appear to the left in exactly one clause, so $R$ will indeed be a GNF theory.\\

\textbf{Example: }
\begin{align}
  \text{CNF:\quad}& (a \vee b \vee \neg c) \wedge (\neg d \vee e) \wedge f\\
  \text{GNF:\quad}& c' \leftrightarrow \neg c, d' \leftrightarrow \neg d, y_1 \leftrightarrow (\neg a \wedge \neg b \wedge \neg c'), y_2 \leftrightarrow (\neg d' \wedge \neg e), y_3 \leftrightarrow \neg f
\end{align}
