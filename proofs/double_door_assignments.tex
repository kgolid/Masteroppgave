% !TEX root= ../main.tex
\section{Provability of NAND-clauses in Figure~\ref{fig:double_open_door}}
\label{sec:Provability of NAND-clauses in double door}
Like our previous table in Figure~\ref{fig:v3_counter_table}, the following table shows what pair of vertices from Figure~\ref{fig:double_open_door} can be 1 under the same assignment and thus correspond to a binary NAND-clause unprovable in Neg.
\begin{figure}[!h]
  \centering
  \[\begin{array}{|c||c|c|c|c|c|c|c|c|c|c|c|c|c|c|c|c|c|c|}
    \hline
          & a & b^L & x^L_1 & x^L_2 & y^L_1 & y^L_2 & c^L_1 & c^L_2 & b^R & x^R_1 & x^R_2 & y^R_1 & y^R_2 & c^R_1 & c^R_2 & t\\ \hline\hline
    a     & & & & & & & & & & & & & & & & \\ \hline
    b^L   &-& X & X & X & & & X & X & X & X & X & X & X & X & X & \\ \hline
    x^L_1 &-&-& X & & & X & & X & X & X & X & X & X & X & X & \\ \hline
    x^L_2 &-&-&-& X & X & & X & & X & X & X & X & X & X & X & \\ \hline
    y^L_1 &-&-&-&-& X & X & X & X & X & X & X & & & X & X & \\ \hline
    y^L_2 &-&-&-&-&-& X & X & X & X & X & X & & & X & X & \\ \hline
    c^L_1 &-&-&-&-&-&-& X & & X & X & X & X & X & X & X & \\ \hline
    c^L_2 &-&-&-&-&-&-&-& X & X & X & X & X & X & X & X & \\ \hline
    b^R   &-&-&-&-&-&-&-&-& X & X & X & & & X & X & \\ \hline
    x^R_1 &-&-&-&-&-&-&-&-&-& X & & & X & & X & \\ \hline
    x^R_2 &-&-&-&-&-&-&-&-&-&-& X & X & & X & & \\ \hline
    y^R_1 &-&-&-&-&-&-&-&-&-&-&-& X & X & X & X & \\ \hline
    y^R_2 &-&-&-&-&-&-&-&-&-&-&-&-& X & X & X & \\ \hline
    c^R_1 &-&-&-&-&-&-&-&-&-&-&-&-&-& X & & \\ \hline
    c^R_2 &-&-&-&-&-&-&-&-&-&-&-&-&-&-& X & \\ \hline
    t     &-&-&-&-&-&-&-&-&-&-&-&-&-&-&-& \\ \hline
  \end{array}\]
  \caption{}
  \label{fig:double_door_counter_table}
\end{figure}
