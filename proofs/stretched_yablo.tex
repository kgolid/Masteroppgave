% !TEX root= ../main.tex
\section{Inconsistency of Stretched Yablo}
\label{sec:Inconsistency of Stretched Yablo}
One variation of the Yablo-graph is the Stretched Yablo-graph, presented by Walicki in \cite{michal-completeness}.
While the Yablo-graph is a ray where each vertex on it has direct edges to all the vertices after it, the Stretched Yablo-graph is a ray where each vertex has \textit{disjoint paths} (with certain properties depending on the variation of Stretched Yablo) to all vertices in after it.

The variation on which we will be proving an inconsistency is one with a set of core vertices $\mathbb{N}$ where each vertex $x$ has a disjoint path to every vertex $y$ after it, such that the length of the path is $(2 \times (y - x) - 1)$.

\begin{figure}[!h]
  \centering
  \begin{tikzpicture}
    [
    point/.style={circle,draw,inner sep=0pt,minimum size=2mm},
    collection/.style={thick,rectangle,draw,inner sep=0pt,minimum height=14mm, minimum width= 9mm}
    ]
	  \matrix (m) [matrix of math nodes,row sep=3em,column sep=1em,minimum width=2em]
	  {   & & 1^1_4 & 2^1_4 & 3^1_4 & 4^1_4 & & 1^2_5 & 2^2_5 & 3^2_5 & 4^2_5 & & \\
	  	1 & & & 2 & & & 3 & & & 4 & & & 5 \\
	  	  & & 1^1_3 & & 2^1_3 & 1^2_4 & & 2^2_4 & 1^3_5 & & 2^3_5 & & \\
		  & 1^1_5 & & 2^1_5 & & 3^1_5 & & 4^1_5 & & 5^1_5 & & 6^1_5 & \\
		};
	  \path[-stealth]
	  	(m-1-3) edge (m-1-4)
		(m-1-4) edge (m-1-5)
		(m-1-5) edge (m-1-6)
		(m-1-6) edge (m-2-10)
		(m-1-8) edge (m-1-9)
		(m-1-9) edge (m-1-10)
		(m-1-10) edge (m-1-11)
		(m-1-11) edge (m-2-13)
	  	(m-2-1) edge [line width=1pt] (m-2-4)
				edge (m-3-3)
				edge (m-1-3)
				edge (m-4-2)
		(m-2-4) edge [line width=1pt] (m-2-7)
				edge (m-3-6)
				edge (m-1-8)
		(m-2-7) edge [line width=1pt] (m-2-10)
				edge (m-3-9)
		(m-2-10) edge [line width=1pt] (m-2-13)
		(m-3-3) edge (m-3-5)
		(m-3-5) edge (m-2-7)
		(m-3-6) edge (m-3-8)
		(m-3-8) edge (m-2-10)
		(m-3-9) edge (m-3-11)
		(m-3-11) edge (m-2-13)
		(m-4-2) edge (m-4-4)
		(m-4-4) edge (m-4-6)
		(m-4-6) edge (m-4-8)
		(m-4-8) edge (m-4-10)
		(m-4-10) edge (m-4-12)
		(m-4-12) edge (m-2-13)
	  ;
	\end{tikzpicture}
  \caption{Stretched Yablo}
  \label{fig:stretched_yablo}
\end{figure}
Shown above are parts of the Stretched Yablo-graph described.
We denote the core vertices by natural numbers.
The peripheral vertices contained in each path is denoted by $n^x_y$ where $x$ and $y$ is the source and target, respectively, of the path in which the vertex is contained.
$n$ denoted the relative position on the path.

Note that since two core vertices $x$ and $y$ have a peripheral path between them of length $L^x_y = 2(y-x) - 1$ (from the definition above), each path consists of $L^x_y - 1 = 2(y-x) - 2$ peripheral vertices.
The vertex pointing back onto the core ray will thus alway be the vertex $(2(y-x)-2)^x_y$.

Based on the given definition of Stretched Yablo and the notation from above, we get the following sets of axioms:
\begin{align}
  N1a &= \{\; \ol{x(x+1)} \;|\; x \in \mathbb{N} \; \} \\
	N1b &= \{\; \ol{x1^x_y} \;|\; x,y \in \mathbb{N}, x+1 < y \; \} \\
	N2 &= \{\; \ol{n^x_y(n+1)^x_y} \;|\; n,x,y \in \mathbb{N}, x+1 < y, n < 2(y-x) - 2 \; \} \\
	N3 &= \{\; \ol{n^x_yy} \;|\; n,x,y \in \mathbb{N}, x+1 < y, n = 2(y-x)-2 \; \} \\
	O1 &= \{\; x(x+1)1^x_{x+2}1^x_{x+3}\dots \;|\; x \in \mathbb{N} \; \} \\
  O2 &= \{\; n^x_y(n+1)^x_y \;|\; n,x,y \in \mathbb{N}, x+1 < y, n < 2(y-x) - 2 \; \} \\
	O3 &= \{\; (n^x_yy \;|\; n,x,y \in \mathbb{N}, x+1 < y, n = 2(y-x)-2 \; \}
\end{align}
\begin{align}
  \text{NAND} = N1a \cup N1b \cup N2 \cup N3 \quad\quad\quad \text{OR} = O1 \cup O2 \cup O3
\end{align}
One can identify the different axioms by looking at the figure above\footnote{recall how NAND-clauses corresponds to edges, while OR-clauses corresponds to vertices}:
$N1a$ are the edges making up the core ray of the graph, $N1b$ are the edges going from a core vertex onto a peripheral vertex, $N2$ are the edges between two peripheral vertices and $N3$ are the edges going from a peripheral vertex back onto a core vertex.
$O1$ are the vertices on the core ray, $O2$ and $O3$ are the peripheral vertices, $O3$ being the ones that points back to a core vertex.

In order to prove $\varnothing$ from these axiom, we first prove three sets of intermediate clauses:
\begin{align}
  D1 &= \{\; \ol{xy} \;|\; x,y \in \mathbb{N}, x \neq y \; \} \\
  D1b &= \{\; \ol{1^x_yn^x_y} \;|\; x,y \in \mathbb{N}, x+1 < y, n = 2(y-x)-2 \; \} \\
  D2 &= \{\; \ol{x1^y_z} \;|\; x,y,z \in \mathbb{N}, x \neq z, y+1 < z \; \} \\
  D3 &= \{\; \ol{1^1_x1^y_z} \;|\; x,y,z \in \mathbb{N}, x \neq z, y+1 < z \; \}
\end{align}
We already have $\ol{xy}$ from our axioms whenever $y = x \pm 1$, so let us assume that either $y < x-1$ or $x + 1 < y$.
In both these cases, we get that there exists a peripheral path between $x$ and $y$.
Let us -- without loss in generality -- assume that $x + 1 < y$.
We thus have a path from $x$ to $y$ containing $n = 2 \times (y - x) - 2$ peripheral vertices.
Consider now the following proof:\par
\begin{figure}[!h]
  \centering
  \begin{prooftree*}
  		\Hypo{(N1b)}
  		\Infer[no rule]1{\ol{x1^x_y}}
  		\Hypo{(N2)}
  		\Infer[no rule]1{\ol{2^x_y3^x_y}}
  		\Infer[left label = (O2) $1^x_y2^x_y$]2{\ol{x3^x_y}}
      \Hypo{(N2)}
  		\Infer[no rule]1{\ol{4^x_y5^x_y}}
  		\Infer[left label = (O2) $3^x_y4^x_y$]2{\ol{x5^x_y}}
      \Hypo{(N2)}
  		\Infer[no rule]1{\ol{6^x_y7^x_y}}
  		\Infer[left label = (O2) $5^x_y6^x_y$]2{\ol{x7^x_y}}
  		\Ellipsis{}{\ol{x(n-1)^x_y}}
  		\Hypo{(N3)}
  		\Infer[no rule]1{\ol{n^x_yy}}
  		\Infer[left label = (O2) $(n-1)^x_y n^x_y$]2{\ol{xy}}
  \end{prooftree*}
  \caption{Proof, D1}
  \label{fig:proof_d1}
\end{figure}
\FloatBarrier
Intuitively, this proof follows along the peripheral path between the two core vertices.
It is important to notice that since all our peripheral paths are odd in length, $(n-1)$ is always odd.
This guarantees that we are able to prove $\ol{x(n-1)^x_y}$ in $n/2$ steps using the strategy above.

Using $x,y,n$ from above, the proof of D1b follows a similar pattern as D1:\par
\begin{figure}[!h]
  \centering
  \begin{prooftree*}
  		\Hypo{(N2)}
  		\Infer[no rule]1{\ol{1^x_y2^x_y}}
  		\Hypo{(N2)}
  		\Infer[no rule]1{\ol{3^x_y4^x_y}}
  		\Infer[left label = (O2) $2^x_y3^x_y$]2{\ol{1^x_y4^x_y}}
      \Hypo{(N2)}
  		\Infer[no rule]1{\ol{5^x_y6^x_y}}
  		\Infer[left label = (O2) $4^x_y5^x_y$]2{\ol{1^x_y6^x_y}}
      \Hypo{(N2)}
  		\Infer[no rule]1{\ol{7^x_y8^x_y}}
  		\Infer[left label = (O2) $6^x_y7^x_y$]2{\ol{1^x_y8^x_y}}
  		\Ellipsis{}{\ol{1^x_y(n-2)^x_y}}
  		\Hypo{(N2)}
  		\Infer[no rule]1{\ol{(n-1)^x_yn^x_y}}
  		\Infer[left label = (O2) $(n-2)^x_y (n-1)^x_y$]2{\ol{1^x_yn^x_y}}
  \end{prooftree*}
  \caption{Proof D1b}
  \label{fig:proof_d1b}
\end{figure}
Just like with D1, it is crucial that the path is odd in order for this proof strategy to work.
Again, we have a trivial case where $x = y$ making the clause simply an instance of $N1b$.
We therefore continue under the assumption that $x \neq y$, together with the other restrictions from the definition of $D2$.
Using the same notation as above, where $n$ is the number of vertices in the peripheral path between $y$ and $z$, we now get the following short proof:
\begin{prooftree*}
  \Hypo{(D1)}
  \Infer[no rule]1{\ol{xz}}
  \Hypo{(D1b)}
  \Infer[no rule]1{\ol{1^y_zn^y_z}}
  \Infer[left label = (O3) $n^y_zz$]2{\ol{x1^y_z}}
\end{prooftree*}
Using $D1$ and $D1b$ in the above proof, restricts us to cases where $x \neq z$ and where $y+1 < z$, but these are restrictions we have already assumed.
$n$ now denotes the number of vertices in the peripheral path between $y$ and $z$.
\begin{prooftree*}
  \Hypo{(D2)}
  \Infer[no rule]1{\ol{1^1_xz}}
  \Hypo{(D1b)}
  \Infer[no rule]1{\ol{1^y_zn^y_z}}
  \Infer[left label = (O3) $n^y_zz$]2{\ol{1^1_x1^y_z}}
\end{prooftree*}
We are again restricting our cases by using $D3$ and $D1b$, but these restrictions are already assumed.
We will prove the inconsistency of Stretched Yablo by first proving $\ol{1},\ol{2},\ol{1^1_3},\ol{1^1_4}\ol{1^1_5}\dots$.

Proving $\ol{1}$:
\begin{prooftree*}
  \Hypo{(N1)}
  \Infer[no rule]1{\ol{12}}
  \Hypo{(D1)}
  \Infer[no rule]1{\ol{13}}
  \Hypo{(D2)}
  \Infer[no rule]1{\ol{11^2_4}}
  \Hypo{(D2)}
  \Infer[no rule]1{\ol{11^2_5}}
  \Hypo{(D2)}
  \Infer[no rule]1{\ol{11^2_6}}
  \Hypo{\dots}
  \Infer[left label = (O1) $2 3 1^2_4 1^2_5 1^2_6\dots$]6{\ol{1}}
\end{prooftree*}

Proving $\ol{2}$:
\begin{prooftree*}
  \Hypo{(N1)}
  \Infer[no rule]1{\ol{23}}
  \Hypo{(D1)}
  \Infer[no rule]1{\ol{24}}
  \Hypo{(D2)}
  \Infer[no rule]1{\ol{11^3_5}}
  \Hypo{(D2)}
  \Infer[no rule]1{\ol{11^3_6}}
  \Hypo{(D2)}
  \Infer[no rule]1{\ol{11^3_7}}
  \Hypo{\dots}
  \Infer[left label = (O1) $3 4 1^3_5 1^3_6 1^3_7\dots$]6{\ol{2}}
\end{prooftree*}

Proving $\ol{1^1_3}$:
\begin{prooftree*}
  \Hypo{(D2)}
  \Infer[no rule]1{\ol{1^1_34}}
  \Hypo{(D2)}
  \Infer[no rule]1{\ol{1^1_35}}
  \Hypo{(D3)}
  \Infer[no rule]1{\ol{1^1_31^4_6}}
  \Hypo{(D3)}
  \Infer[no rule]1{\ol{1^1_31^4_7}}
  \Hypo{(D3)}
  \Infer[no rule]1{\ol{1^1_31^4_8}}
  \Hypo{\dots}
  \Infer[left label = (O1) $4 5 1^4_6 1^4_7 1^4_8\dots$]6{\ol{1^1_3}}
\end{prooftree*}

Proving $\ol{1^1_4}$:
\begin{prooftree*}
  \Hypo{(D2)}
  \Infer[no rule]1{\ol{1^1_45}}
  \Hypo{(D2)}
  \Infer[no rule]1{\ol{1^1_46}}
  \Hypo{(D3)}
  \Infer[no rule]1{\ol{1^1_31^5_7}}
  \Hypo{(D3)}
  \Infer[no rule]1{\ol{1^1_31^5_8}}
  \Hypo{(D3)}
  \Infer[no rule]1{\ol{1^1_31^6_9}}
  \Hypo{\dots}
  \Infer[left label = (O1) $5 6 1^5_7 1^5_8 1^5_9\dots$]6{\ol{1^1_4}}
\end{prooftree*}

The clauses $\ol{1^1_4},\,\ol{1^1_5},\,\ol{1^1_5},\,\ol{1^1_7},\,\dots$ can be proven in exactly the same manner as above.
With the results from above we can finally prove $\varnothing$:
\begin{prooftree*}
  \Hypo{\ol{1}}
  \Hypo{\ol{2}}
  \Hypo{\ol{1^1_3}}
  \Hypo{\ol{1^1_4}}
  \Hypo{\ol{1^1_5}}
  \Hypo{\dots}
  \Infer[left label = (O1) $1 2 1^1_3 1^1_4 1^1_5\dots$]6{\varnothing}
\end{prooftree*}

Having proved $\varnothing$ in Neg, we get from soundness of Neg that our Stretched Yablo-graph is indeed inconsistent.
