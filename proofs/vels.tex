% !TEX root= ../main.tex
\externaldocument{../binary_nands/odd_vels}
\section{Provability of NAND-clauses from vels}
\label{sec:Provability of NAND-clauses from vels}
This section will prove the following statement:
Given a graph where two vertices $a$ and $b$ are connected by a vel, as defined in Section~\ref{sec:Odd vels}, the binary NAND-clause $\ol{ab}$ is provable in Neg.

Let $\mathbf{G} = (G,N)$ be a graph containing the vertices $a,b$ such that they have a vel between them.
By definition, this means that there exists a vertex $c$ such that there is an oddly trimmed path from $a$ to $c$ and from $b$ to $c$, one of odd length and one of even length.

Let $P$ and $Q$ be the two sets containing the vertices in the path from $a$ to $c$ and from $b$ to $c$, respectively.
We will denote each element of $P$ as $p_i$ where $i \in \mathbb{N}$ is the position of that vertex in the path, starting at 0.
The elements of $Q$ will be named $q_i$ by the same rule.
We immediately have that $a = p_0$ and $b = q_0$.
As long as the trimming restrictions are met, $P$ and $Q$ might overlap, i.e. we might have cases where $p_j = q_k$ from some $i$ and $j$.

We assume, without any loss of generality, that the path from $a$ to $c$ is of odd length, making the path from $b$ to $c$.
We denote the lengths of the two paths by the numbers $n$ and $m$, giving us that $c = p_n = q_m$.
The path from $b$ to $c$ might also be empty, making $b = p_0 = c$.

\begin{figure}[!h]
  \centering
  \begin{tikzpicture}
    [
    point/.style={circle,draw,inner sep=0pt,minimum size=2mm}
    ]
    \node (a) at (0,5) [point, label=left:${a = p_0}$] {};
    \node (p1) at (1,4) [point, label=left:$p_1$] {};
    \node (pd) at (2,3) [] {$\dots$};
    \node (pn2) at (3,2) [point, label=left:$p_{n-2}$] {};
    \node (pn1) at (4,1) [point, label=left:$p_{n-1}$] {};
    \node (c) at (5,0) [point, label=right:${c = p_n = q_m}$] {};
    \node (qm1) at (6,1) [point, label=right:$q_{m-1}$] {};
    \node (qm2) at (7,2) [point, label=right:$q_{m-2}$] {};
    \node (qd) at (8,3) [] {$\dots$};
    \node (q1) at (9,4) [point, label=right:$q_1$] {};
    \node (b) at (10,5) [point, label=right:${b = q_0}$] {};
    \draw [-latex] (a) to (p1);
    \draw [-latex] (p1) to (pd);
    \draw [-latex] (pd) to (pn2);
    \draw [-latex] (pn2) to (pn1);
    \draw [-latex] (pn1) to (c);
    \draw [-latex] (b) to (q1);
    \draw [-latex] (q1) to (qd);
    \draw [-latex] (qd) to (qm2);
    \draw [-latex] (qm2) to (qm1);
    \draw [-latex] (qm1) to (c);

    \node (e1) [below left=6mm and 4mm of c]  {};
    \node (e2) [below=8mm of c] {};
    \node (e3) [below right=6mm and 4mm of c] {};
    \draw [dashed] (e1) to (c);
    \draw [dashed] (e2) to (c);
    \draw [dashed] (e3) to (c);

    \node (ba) [below left=6mm and 4mm of a] {};
    \node (bn1) [below left=6mm and 4mm of pn1] {};
    \node (bb) [below right=6mm and 4mm of b] {};
    \node (bm2) [below right=6mm and 4mm of qm2] {};
    \draw [dashed] (ba) to (a);
    \draw [dashed] (bn1) to (pn1);
    \draw [dashed] (bb) to (b);
    \draw [dashed] (bm2) to (qm2);
  \end{tikzpicture}
  \caption{A general vel between $a$ and $b$}
  \label{fig:general_odd_vel}
\end{figure}

Let $\text{NAND}_G$ and $\text{OR}_G$ denote the axiomatic NAND- and OR-clauses we get from our graph.
We can work out the following subset of $\text{NAND}_G$:
\begin{align}
  \text{NAND}_V = \{\; \ol{p_ip_{i+1}} \;|\; 0 \leq i < n\;\} \;\cup\; \{\; \ol{q_jq_{j+1}} \;|\; 0 \leq j < m\;\}
\end{align}
Since both paths are oddly trimmed, every vertex at an odd position in its path will only have one out-edge, resulting in a binary OR-clause.
This makes us able to work out the following subset of $\text{NAND}_V$:
\begin{align}
  \text{OR}_V = \{\; p_ip_{i+1} \;|\; 0 \leq i < n, i\text{ is odd}\;\} \;\cup\; \{\; q_jq_{j+1} \;|\; 0 \leq j < m, j\text{ is odd}\;\}
\end{align}
The proof of $\ol{ab}$ can now be worked, only based on the axioms $\Gamma_G = \text{NAND}_G \cup \text{OR}_G$:
